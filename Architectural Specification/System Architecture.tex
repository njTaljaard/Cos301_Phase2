\documentclass[12pt]{article}

\usepackage[english]{babel}
\usepackage[utf8x]{inputenc}
\usepackage{amsmath}
\usepackage{graphicx}
\usepackage{listings}
\usepackage{color}
\usepackage{pdfpages}
\vfuzz=3000pt
\title{Software Architecture Specification}

%Remove the author and date fields and the space associated with them
% from the definition of maketitle!
\makeatletter
\renewcommand{\@maketitle}
{
	\newpage
 	\null
 	\vskip 0em%
 	\begin{center}%
  	{\huge \bf \@title \par}%
 	\end{center}%
 	\par
} 
\makeatother

\begin{document}

%**************************************************************************
%  FRONT PAGE
%**************************************************************************

\maketitle

\vspace{4em}

\begin{center}%

  \LARGE {\bf Group 4}\\[2em]
  \LARGE {\bf Group Members:}\\[1em]
  \large
      Nico Taljaard			(10153285)	\\
      Neels van Rooyen		(29052735)	\\
	  Stephan Viljoen       (11008408)  \\[6em]
      
      {\bf Version 1.0}
    
\end{center}%

\newpage

%**************************************************************************
%  CHANGELOG
%**************************************************************************

\begin{tabbing}
\hspace*{3cm}\=\hspace*{3cm}\=\hspace*{8cm}\=\hspace*{3cm} \kill
03/03/2013 \> Version 1.0 \> Document Created \> Nico Taljaard\\

\end{tabbing}

\newpage

%**************************************************************************
%  TABLE OF CONTENTS
%**************************************************************************

\tableofcontents

\newpage

%**************************************************************************
%  INTRODUCTION
%**************************************************************************

\section{Introduction}

	\subsection{Purpose}


	\subsection{Document Conventions}


	\subsection{Project Scope}


	\subsection{References}


	\subsection{Related Documents}



%**************************************************************************
%  SYSTEM DESCRIPTION
%**************************************************************************
\section{System Description}

	\subsection{Technologies}
	These software architecture design decisions are based on the constraints defined by the master requirement specification document:
\indent\indent \linebreak\linebreak\textbf{Programming Languages}
\begin{itemize}
\item Python
\begin{itemize}
\item A high-level object oriented programming language that can be used for scripting,prototyping, testing and very easy to learn and implement.
\end{itemize}
\item HTML
\begin{itemize}
\item The system must be accessible to users and HTML Markup language will be used to for 	creating structured displayable content in a web browsers (Mozilla Firefox, Google Chrome, Apple Safari and Microsoft Internet Explorer). Reference 4.1.1.
\end{itemize}
\item CSS
\begin{itemize}
\item Used for formatting of HTML documents creating rich web interfaces.
\end{itemize}
\item JavaScript
\begin{itemize}
\item Dynamic programming language used for client side interaction.
\end{itemize}
\item SQL
\begin{itemize}
\item Used for data managing (insert, delete, update and read) of your relational data using the Object-Relational Mapper.
\end{itemize}
\item Java
\begin{itemize}
\item A Powerful language with vast libraries with which the android interface will be created. Reference 4.1.2
\end{itemize}
\end{itemize}
\indent\indent \linebreak\linebreak\textbf{Application Server}
\begin{itemize}
\item A Django application server will be used to run and deploy the system within the cs.up.ac.za Apache web server. The webserver must be published as SOAP –based.

\end{itemize}                                
\indent\indent\linebreak \linebreak\textbf{Database Technology}
\begin{itemize}
\item MySQL implement the quality requirements stated by the master specification and it decouples the system from the database.
\begin{itemize}
\item Scalability and Flexibility
\begin{itemize}
\item Provides scalability, sporting the capacity to handle deeply embedded applications 
Platform flexibility (Linux, UNIX, Mac and Windows)
\end{itemize}
\item Security 
\begin{itemize}
\item  Offering exceptional features that ensure absolute data protection when looking at database authentication, backup and recovery.
\end{itemize}
\end{itemize}
\end{itemize} 
	
	\subsection{Frameworks}
\indent\indent \linebreak\linebreak\textbf{Object-Relational Mapper}
\begin{itemize}
\item A technique for converting data between incompatible types in object-oriented languages. With our application data to outlive the applications process. 
The persistence to the relational database must be done using the Object-Relational Mapper bundled with Django. Django is the chosen Object-Relational mapper used to obtain, retrieve and transfer data between a variety of platforms such as from the MySQL database to the web client and android interface. You can define your data model in Python and access through the database API.
\end{itemize}

\indent\indent \linebreak\linebreak\textbf{Web Service Frame Work}
\begin{itemize}
\item The Django framework is based on the model view controller, which is used for creating complex database driven websites with the ability of reusability of components and rapid development. 
The key quality requirements that Django accommodate are scalability, security, reliability and integration at a lesser extent.

\end{itemize}	
	
	\subsection{Protocols}
	
	
	\subsection{Libraries}


%**************************************************************************
%  OVERALL ARCHITECTURE
%**************************************************************************
\section{Overall Architecture}



%**************************************************************************
%  ARCHITECTUAL STRATEGIES
%**************************************************************************
	\subsection{Architectural Strategies}
	
	

	\subsubsection {Thread Pooling:} 


	\subsubsection {Clustering:} 


	\subsubsection{Interception:}
	

	\subsubsection {Run-Time Lookups:}


	\subsubsection {Queuing:}


%**************************************************************************
%  ARCHITECTUAL PATTERNS
%**************************************************************************

	\subsection{Architectural Patterns}


	\subsubsection{Layering:} 



	\subsubsection{Model-View-Controller(MVC:}



	\subsubsection{Pipes and Filters:}



	\subsection{Reference Architectures}



	\subsection{Technology and Framework Selection}



	\subsubsection{Database Servers}



	\subsubsection{Web Server}



%**************************************************************************
%  DETAILS OF SYSTEM
%**************************************************************************
\section{Details of System}

	\subsection {Logic View}
	
	

	\subsubsection{Class Diagram}



	\subsubsection{Communication Diagram}



	\subsubsection {Sequence Diagram}



	\subsection{Development View}



	\subsubsection{Package Diagram}



	\subsubsection{Component Diagram}



	\subsection{Process View}



	\subsection {Physical View}


	
	\subsubsection{Deployment Diagram}



	\subsection{Scenarios}

	\subsubsection{Use Cases Diagram}


	\subsection {Component Diagram}


	\section{Traceability Matrix}



%**************************************************************************
%  POLICIES
%**************************************************************************
\lstset{language=Java}

\section{Policies}


	\subsection{Performance}


	
	\subsection{Capacity and Scalability}



	
	\subsection{Availability}


	
	\subsection{Maintainability}



	\subsection{Recovery}



\section{Glossary}



\end{document}